\documentclass[UTF8]{ctexart}
\usepackage{cite}
\usepackage{url}
\usepackage{amsfonts}    
\usepackage{amsmath}      
\usepackage{amssymb}
\usepackage{listings}
\usepackage{graphicx}
\usepackage{subfigure}
\usepackage{xspace}
\usepackage{float} 
\usepackage{markdown}

\title{读书笔记(四)}
\author{黄怀宇}
\date{\today}
\lstdefinestyle{styleM}{language=matlab}
\lstdefinestyle{stylePy}{language=python}

\begin{document}
\maketitle

\section{各个工具的特点和原理(续)}

\subsection{doc2vec}

\subsection{word2vec}

\subsection{one-hotm}

\subsection{BOW}

\subsection{CBOW}

\subsection{Skip-Gram}

\section{工作进度}
\subsection{前后端数据传输}
数据初步处理后利用 CORS 实现跨域数据传输。想通过 node.js 和 cors 通过 POST 传输数据实现。
\subsection{基于业务特点的设计}
微博的评论是随着页面滚动而加载出来的,数据是实时生成的,服务端的 js 脚本应该实时增量式地将文本发给服务器处理,服务器根据已经训练好的算法实时返回(判断刚加载出来的评论是否应该删除)。

中途不需要人工操作。

\subsection{服务端算法模型训练}
需要爬取尽可能多的微博评论,进行无监督聚类,将训练好了之后的模型用于判断新来的评论属于哪个类别。

\subsection {第二阶段再做}
加上人工优化,持续反馈

\subsection {还需做的}
1. 预先加载评论好(滚动滑条就会加载很多内容,或其它方法)再返回,因为大部分都被删除了。脚本优化。

2. 爬取微博尽可能多的评论,选择合适的算法组合进行聚类。
\subsection {项目难点}


\bibliography{cites.bib}
\bibliographystyle{ieeetr}

\end{document}